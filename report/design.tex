% ================================= chapter 3 ================================= %
\chapter{制御系設計}
\subsection{特性解析}
第\ref{chapter_modeling}章で物理パラメータの決定と数式モデルの導出を行った。
ここでは、決定したパラメータから、式(\ref{linear_k}),式(\ref{linear_general})の特性解析を行う。
倒立振子の線形モデルの状態空間表現は以下のようになる。

\begin{equation}
    A,\ B,\ Cの値
    \label{ABC}
\end{equation}

まず、行列$A$の安定判別を行う。$A$のすべての固有値に関して、実部が負であれば$A$は安定性行列である。
すなわち、$A$の固有値$\lambda_{1}$, $\lambda_{2}$, $...$に対し、

$$
    \mbox{Re}[\lambda_{i}] < 0
$$

が成立すればよい。\MaTX{}で$A$の固有値を算出した結果を表\ref{eigen_A}に示す。

\begin{table}[htbp]
    \begin{center}
        \caption{表\ref{eigen_A}: $A$の固有値}
        \begin{tabular}{|c|c|} \hline
            固有値 & Re[$\lambda_{i}$] \\ \hline \hline
            0 & 0 \\ \hline
            0 & 0 \\ \hline
            0 & 0 \\ \hline
            0 & 0 \\ \hline
        \end{tabular}
        \label{eigen_A}
    \end{center}
\end{table}

表\ref{eigen_A}から、倒立振子系のシステムは不安定である。次に、システムの可制御性、可観測性を判別する。
可制御性は、$n$をシステムの次数(倒立振子系のシステムの次数は$n = 4$)に対し、可制御性行列、

$$
    U_{C} =
    \left[
        \begin{array}{ccccc}
            B  &  AB  &  A^2B  &  \dots  &  A^{n-1}B
        \end{array}
    \right]
$$

のランクがシステムの次数と等しければ満たされる。また、可観測性行列は、

$$
    U_{o} = 
    \left[
        \begin{array}{c}
            C \\
            CA \\
            CA^2 \\
            \vdots \\
            CA^{n-1}
        \end{array}
    \right]
$$

であり、$U_{o}$のランクがシステムの次数と等しければ可観測となる。
式(\ref{ABC})において、\MaTX{}を用いて$U_{c}$, $U_{o}$のランクを計算した結果、以下のようになった。

$$
    Uc, Uoのランク
$$

以上から、倒立振子系のシステムは不安定であり、可制御、可観測である。

\subsection{制御システムの構成}
図



\subsection{状態フィードバック$F$の設計}

\subsection{$\hat{A}$, $\hat{B}$, $\hat{C}$, $\hat{D}$, $\hat{J}$の設計}

\subsection{制御システムの離散化}

\subsection{振り上げ制御と安定化}



% =============================== chapter 3 END =============================== %