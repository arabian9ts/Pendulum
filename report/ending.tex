% ================================= chapter 6 ================================= %
\chapter{おわりに}
\subsection{まとめ}
今回の実験を通して,制御系のモデリングから制御器の設計,設計した制御器を用いたシミュレーション,
実験を行い,制御の一連の流れを体験することができた.また,制御器を表現するための方法として,Java,
\MaTX{}, Jamoxの3通りの手段を用い,制御系CADの特徴や,実際にプログラムを書くこととの違いを
理解することができた.シミュレーションでは意図した通りに動作するが,実験結果は異なる,あるいは逆の
挙動を示すこともあり,シミュレーションと実験は異なることを実感した.本実験を通して理解した古賀研究室の
プロダクトの特徴,制御の経験をこれからの研究に役立てたい.

\begin{thebibliography}{2}
 \bibitem{1}B.倒立振子の安定化制御,古賀雅伸
 \bibitem{2}制御工学実験第3 倒立振子の安定化制御,古賀雅伸
\end{thebibliography}


\chapter{付録 プログラム}

\section{目標値変更における安定化制御}
\lstinputlisting[caption=xref.mm,label=ref]{xref.mm}

\section{振り上げ制御}
\lstinputlisting[caption=swingup.mm,label=swingup]{swingup.mm}

% =============================== chapter 6 END =============================== %
