% ================================= chapter 6 ================================= %
\chapter{おわりに}
\subsection{まとめ}
今回の実験を通して、制御系のモデリングから制御器の設計、設計した制御器を用いたシミュレーション、
実験を行い、制御の一連の流れを体験することができた。また、制御器を表現するための方法として、Java,
\MaTX{}, Jamoxの3通りの手段を用い、制御系CADの特徴や、実際にプログラムを書くこととの違いを
理解することができた。シミュレーションでは意図した通りに動作するが、実験結果は異なる、あるいは逆の
挙動を示すこともあり、シミュレーションと実験は異なることを実感した。本実験を通して理解した古賀研究室の
プロダクトの特徴、制御の経験をこれからの研究に役立てたい。

% =============================== chapter 6 END =============================== %